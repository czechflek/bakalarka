
\section{Similar solutions}

	\subsection{Parallel Kingdom - Age of Ascension}
	This game was on market for 8 years (2008-2016). Parallel Kingdom is a closest solution to ours.
	
	
	”Parallel Kingdom is a mobile, location based, massively multiplayer game that uses GPS
	location and Google Maps to place users in a virtual world. Parallel Kingdom is the first
	location based RPG for the iOS and Android platforms. The game is set in a virtual world
	or ”Parallel Kingdom” where users claim their territories based on their GPS location or by making friends who invite them to travel to new places. Parallel Kingdom is a freemium
	game and utilizes a virtual goods revenue model.”
	
	\subsection{Ingress}
	Developed by Niantic, which was then part of Google, this game was released in 2013 for
	Android and in 2014 for iOS.[4] It is a location based, massively multiplayer game. A player have to choose one of the two factions, Enlightened or Resistance, and then as a part of his 	team capture regions of the game map. A faith of each faction relies on players’ cooperation. Thanks to that players meet in real life and coordinate their actions.
	
	Ingress was the first very successful augmented reality game with more than 10 000 000
	installs.
	
	\subsection{Pokémon GO}
	After its success with Ingress, Niantic started working on a new game Pokemon GO. Once
	released, the game became incredible hit. Even though the game faced many problems during
	its launch, mainly caused by the unexpected success and more active users than Pokémon
	GO was able to handle, in the first 80 days Pokémon GO reached about 550 millions downloads and earned about \$470 million.
	
	The game is very similar to Ingress and uses the same crowd-sourced geographical data.
	
\section{Use Cases}

	\subsection{Use case diagram}
	
		\begin{figure}[h]	
			\includegraphics[width=\textwidth]{figures/UseCaseDiagram}
			\centering			
			\caption{Use case diagram}
			\label{fig:usecasediagram}
		\end{figure}
	
	\subsection{Use case descriptions}
		The following section refer to the use-cases introduced in Figure \ref{fig:usecasediagram}.
		\begin{enumerate}
			\item \textbf{Register account} \\
			\item \textbf{Login to the game} \\
			\item \textbf{Change the inventory content} \\
			\item \textbf{Go on a quest} \\
			\item \textbf{Fight a common monster} \\
			\item \textbf{Respawn} \\
			\item \textbf{Update a quest progress} \\
			\item \textbf{Create a new quest} \\
			\item \textbf{Edit or delete a quest} \\
						
		\end{enumerate}
	
	
\section{Requirements}

	\subsection{Functional Requirements}
		\subsubsection{Rules}
		\begin{enumerate}
			\item \textbf{The server must provide API to clients} \\
			The key requirement for the server is to allow receiving HTTP(S) requests. When processed, the server responds in JSON format.
						
			\item \textbf{The player's character has attributes} \\
			The character has a set of attributes, including health, experience, level, and owned gold. The maximum health increases with level. The experience is rewarded after certain actions, e.g. after killing a monster. The gold is primary in-game currency.
			  
			\item \textbf{A player can own items} \\
			A player has an inventory which can contain various types of items. The item can be for example sword, potion, armor etc.
			
			\item \textbf{A game object has a type and inherits all its properties} \\
			The type of game object specifies allowed actions, its attributes, default name and description.			
			
			\item \textbf{A game object can be a monster} \\
			The monster can be killed but can also inflict damage to the player. It has its own inventory and there's a reward for killing the monster in a form of gold and experience.
			
			\item \textbf{A game object can be a shop} \\
			The shop can contain several items with specified price.  
			
			\item \textbf{A game object can be an item} \\	
			The item can be one of the many objects useful to a player. Examples of the items are health potion, sword, armor, necklace and similar.
			
			\item \textbf{Each game object has its own inventory} \\
			The inventory contains other game objects. Example of this requirement is a monster with a potion and a sword in its inventory; both will be given to the player who kills the monster.  
			
			\item \textbf{The server stores a list of predefined locations} \\	
			Real geographic location for the game objects are stored on the server to ensure every player has the same location-object pair. 
								
			\item \textbf{A game object can independently exist at many locations} \\
			This requirement aims to help maintain the game objects efficiently by administrators. It allows creating small set of abstract game objects with predefined inventories and other attributes. 
			
			\item \textbf{If a player kills a monster at a location, the monster will be hidden for a period} \\
			To prevent the player from killing the same monster continuously without a need of moving somewhere else, the location should be hidden for a certain period after the kill.
			
			\item \textbf{The server should provide API for administration} \\	
			
					
			\item \textbf{The server must persist player’s profile between sessions} \\
		\end{enumerate}
		\subsubsection{Features}
		\begin{enumerate}
			\item \textbf{A player registers and logs in the game using Google account} \\
			For player's convenience, a Google account is required to play. The application does not have to store any password. 
			\item \textbf{A client can get nearby game objects based on his location} \\
			
			\item \textbf{A player can kill a monster} \\
			\item \textbf{A player can be killed be a monster} \\
			\item \textbf{A player can collect items from the monster he killed} \\
			\item \textbf{A player can equip an item} \\				
			\item \textbf{A player can buy object from a shop} \\
			\item \textbf{A player can use an item from his inventory} \\
			\item \textbf{A player can purchase in-app product} \\
		\end{enumerate}
		
		
	\subsection{Non-functional}
	
		\begin{enumerate}
			\item \textbf{The data layer consists of a database engine and a caching} \\
			Game Server never accesses database directly. Data are retrieved from the database upon request and then cached. Upper layers communicate exclusively with cache.		
			
			\item \textbf{The server provides an API for client} \\
			The API support at least following:
			\begin{enumerate}
				\item Retrieve nearby game locations
				\item Create player
				\item Update player's data (inventory, experience, resources, quests progress etc.)
				\item Provide information about a quest 			
			\end{enumerate}
			
			\item \textbf{The communication between client and server parts of the application must be secure} \\
			All data sent from and to a client has to be encrypted.
	
			\item \textbf{Client can only connect to a Connection Server} \\
			Several Connection Servers exist to prevent a bottle-neck. Client selects the Connection Server by an algorithm. Client does not have an access to any other part of the server.
		\end{enumerate}
	\subsection{System and Interface}
	
		\begin{enumerate}
			\item \textbf{System uses Java 8 SE as an execution environment} \\
			
			\item \textbf{Operating system for the server is Debian 8} \\
			
			\item \textbf{Database engine is MySQL} \\
			
			\item \textbf{Cache engine is Redis} \\
		\end{enumerate}
	
\section{Technology}

	\subsection{Frameworks}
	