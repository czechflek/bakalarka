This thesis aimed to create a prototype of the server part of a role-playing game with features of augmented reality. I explored and analyzed exiting similar games. After discussion with my colleague Tomáš Zahálka, I defined the rules and features of the game prototype. 

I analyzed use cases and specified requirements to clarify expected server functionality. I explored available solutions for \textit{database management system} and chose t o use \textit{MySQL}. I decided to use several Java frameworks and libraries to handle database communication, JSON serialization, and manage API endpoints. I designed the structure of the server and described actions a user can perform. Based on these actions, I created a specification for public API as well as private one for internal communication among components. I designed and implemented a database model. I obtained and imported initial game locations to the database. In implementation phase, I created all specified API endpoints, implemented game logic a database communication. Administration section was secured and requires authentication. I created an index of location and configured a cache to improve database performance. Lastly I performed unit, system  and stress tests as well as static code analysis.

Since the server is currently in prototype version, many features and further improvements are planed for future development. I plan to improve security, error handling and increase coverage of the unit tests before production release. One of the planned features is a quest system. My design allow easy scalability to support many concurrent users and high extensibility which enables me to create full, market-ready server.