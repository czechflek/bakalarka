\section{Prerequisites}
The server has several dependencies which should be met before the installation process. Even though the target platform is Debian, the server should work on any operating system with Java support. While the server might work with software versions other than the ones I included, the compatibility is not guaranteed. The dependencies can be downloaded for free from the cited websites.

\begin{enumerate}
	\item MySQL 5.7 \cite{mysql}
	\item Redis 3.2 \cite{redis}
	\item Java 8 or OpenJDK 8	
\end{enumerate}

\section{Compilation}
Although I enclosed compiled \textit{JAR} files on the SD card, the source code can be compiled using Maven \cite{maven}. Simply move to the root directory of a component and execute a Maven goal \textit{install} and optionally a goal \textit{clean}:\\
\emph{maven clean install}

Compiled \textit{JAR} binaries and generated HTML documentation can be found in \textit{target/} folder upon successful compilation.

\section{Database Initialization}
The database must be initialized prior running the Database server component. I've created an SQL script which creates the database structure and imports initial data, such as game objects, locations, actions, and so on. The script is located at \textit{/BachelorsServer/Data/init\_database.sql} on the enclosed SD card.

The MySQL server might not include timezone information rendering the application broken. You can import the time zones converting system time zones to SQL using\textit{mysql\_tzinfo\_to\_sql} tool and piping the output to \textit{mysql} program \cite{mysqltimezones}:\\
\emph{mysql\_tzinfo\_to\_sql /usr/share/zoneinfo | mysql -u root mysql}

\section{Configuration}
Default server configuration is available in \textit{config.yml} file which can be found in the root folder of each component on the enclosed SD card. The content of the configuration file is in YAML\footnote{YAML format specification can be found at \url{http://www.yaml.org/spec/1.2/spec.html}} format.The following text describes the meaning of important parameters.

\paragraph*{Connection Server}
\begin{description}
	\item[loginServerUrl] URL of a Login Server instance (example: \textit{http://localhost:8090})
	\item[databaseServerUrl] URL of a Database Server instance (example: \textit{http://localhost:8092})
	\item[server] Jetty web server configuration
		\begin{description}
			\item[applicationConnectors] Protocol and port o which the application listens (example: \textit{HTTP} and \textit{8080})
			\item[adminConnectors] Protocol and port o which the server statistics are available  (example: \textit{HTTP} and \textit{8081})
		\end{description}
\end{description}

\paragraph*{Login Server}
\begin{description}
	\item[mockToken] Substitute for Google ID token used for testing purposes (example: \textit{loremipsum})
	\item[databaseServerUrl] URL of a Database Server instance (example: \textit{http://localhost:8092})
	\item[server] Jetty web server configuration
	\begin{description}
		\item[applicationConnectors] Protocol and port o which the application listens (example: \textit{HTTP} and \textit{8090})
		\item[adminConnectors] Protocol and port o which the server statistics are available  (example: \textit{HTTP} and \textit{8091})
	\end{description}
\end{description}

\paragraph*{Database Server}
\begin{description}
	\item[mysqlUri] Full Java Database Connectivity URI of MySQL server schema where application data are located (example: \textit{jdbc:mysql://localhost/bachelors})
	\item[mysqlUser] User at MySQL server (example: \textit{root})
	\item[mysqlPass] Pasword of the user MySQL user (example: \textit{password})
	\item[redisUri] Full URL of Redis server (example: \textit{localhost})
	\item[redisPort] Optional port of the Redis server (example: \textit{1467})
	\item[redisPass] Password to the Redis server (example: \textit{password})
	\item[server] Jetty web server configuration
	\begin{description}
		\item[applicationConnectors] Protocol and port o which the application listens (example: \textit{HTTP} and \textit{8092})
		\item[adminConnectors] Protocol and port o which the server statistics are available  (example: \textit{HTTP} and \textit{8093})
	\end{description}
\end{description}

\section{Running the Server}
To start the server simply run the compiled JAR files with argument \textit{server} and path to the configuration file. For example:\\
\textit{java -jar ConnectionServer.jar server config.yml}

