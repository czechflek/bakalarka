\section{Unit Testing}
The project use JUnit \cite{junit}. Unit tests are located in \textit{src/test} directory. All components use the test to verify proper serialization and deserialization of JSON requests. Each tested JSON object is defined in \textit{src/test/resources/fixtures}.

The tests are packed in a test suite. This allows running all test by executing \textit{bachelors.connection.api.AllTests}. All tests are passing in the current build.

\section{Static Code Analysis}
I use SonarLint \cite{sonarlint} for on-the fly static code analysis. It offers many useful rules with specified severity and also a description how to fix the issues. Static code analysis discovered many issues, the most notable were:
\begin{itemize}
	\item Hide a private constructor to hide the implicit one in a static class.
	\item Use constant instead of duplicating string literal.
	\item Replace use of System.out by a logger.
	\item Make final constants also static.
\end{itemize}
The static analysis proved to be useful to maintain a good quality of the code and prevent bugs.

\section{System Testing}
I developed a Python script to test most of the endpoints in the real environment. It skips tests of \textit{/login/registration} and \textit{/purchase} as they need valid Google tokens which can't be reused and are not easily obtainable. All tests are passing in the current build.

The script requires Python 3 or newer to run and is located in \textit{BachelorsServer/SystemTest/TestBachelorsPrototype.py}. The tester must initialize the script with values of a tests account in the current database. The script then sends HTTP request to each endpoint testing valid and invalid data.

For example, test \textit{ACTION -- Kill} verifies the client cannot request setting his health to negative value or more than his level allows. Script also tries to kill a monster at a wrong location and vice versa. A successful kill follows a second attempt to kill the same monster which must fail. 


\section{Client Testing}
I tested the application with my colleague using his client part of the game. We didn't discover any bugs which would affect the player in any way.

\section{Stress Testing}
I performed a stress test using Apache JMeter \cite{jmeter}. This testing framework is used to simulate interactions with a web server. Results of this type of test offer useful insight into how many concurrent users can server handle. Hardware specifications of the testing environment are described in Section \ref{section:deploy}.

I created a test plan in a way, that one thread simulates about 10 users. Each thread operates independently for 10 minutes and performs following series of actions:
\begin{enumerate}
	\item Wait 1 second\footnote{The locations are fetched from the server once every 10 seconds in real game client}
	\item Get nearby locations around location A (\textit{/location})
	\item Wait 1 second
	\item Get nearby locations around location B (\textit{/location})
	\item Wait 1 second
	\item Get nearby locations around location C (\textit{/location})
	\item Get profile (\textit{/user})
	\item Get inventory (\textit{/user/inventory})
	\item Go back to step 1
\end{enumerate}

As expected, calls to the endpoint \textit{/location} took the longest time out of the three tested, the difference in latency was about 30\% on average for 500 and 1~500 users. As you can see in Table \ref{tab:loadtestresults}, latency significantly increased at 3~000 users and about 2\% request resulted in an error. At 5 000 users, the server was far beyond its limits and responded with very high latency and about 8\% error rate. This behavior is expected due to the very limited resources of the server. The results might be affected by the performance issues of the test machine, since 500 threads had to be run concurrently to simulate 5 000 users.

Based on the test, server is currently able to handle more than 2~300 concurrent users with no performance issues that would affect clients.

\begin{table}
	\centering
	\begin{tabular}{r || c | c | c | c | c |}
		Users & 500 & 1 500 & 2 300 & 3 000 & 5 000 \\ \hline 
		Latency median [ms] & 108 & 123 & 166 & 473 & 1 737 \\
		Throughput [request/s] & 16.0 & 46.4 & 67.9 & 79.1 & 99.8 \\
	\end{tabular}
	\caption{Result of the server stress test.}
	\label{tab:loadtestresults}
\end{table}	

