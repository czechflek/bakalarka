% options:
% thesis=B bachelor's thesis
% thesis=M master's thesis
% czech thesis in Czech language
% english thesis in English language
% hidelinks remove colour boxes around hyperlinks

\documentclass[thesis=B,english]{FITthesis}[2012/10/20]

%\usepackage[utf8]{inputenc} % LaTeX source encoded as UTF-8
% \usepackage[latin2]{inputenc} % LaTeX source encoded as ISO-8859-2
% \usepackage[cp1250]{inputenc} % LaTeX source encoded as Windows-1250
\usepackage{url}
\usepackage{pdfpages}
\usepackage{float}
\usepackage{graphicx} %graphics files inclusion
\usepackage{fontspec}
\usepackage{enumitem}
% \usepackage{subfig} %subfigures
% \usepackage{amsmath} %advanced maths
% \usepackage{amssymb} %additional math symbols

\usepackage{dirtree} %directory tree visualisation
\usepackage{rotating}
\usepackage[section]{placeins}

\usepackage[
backend=biber,
sortlocale=en_US,
style=iso-numeric
]{biblatex}

\usepackage{footnote}
\makesavenoteenv{tabular}
\makesavenoteenv{table}


\renewcommand{\arraystretch}{1.2}
\renewcommand\labelitemi{---}
\raggedbottom

\setlist[description]{leftmargin=\parindent,labelindent=\parindent}

\bibliography{mybibliographyfile}


% % list of acronyms
% \usepackage[acronym,nonumberlist,toc,numberedsection=autolabel]{glossaries}
% \iflanguage{czech}{\renewcommand*{\acronymname}{Seznam pou{\v z}it{\' y}ch zkratek}}{}
% \makeglossaries

\department{Department of Software Engineering}
\title{Location-based Role Playing Game}
\authorGN{Jakub} %author's given name/names
\authorFN{Čech} %author's surname
\author{Jakub Čech} %author's name without academic degrees
\authorWithDegrees{Jakub Čech} %author's name with academic degrees
\supervisor{Ing. Miroslav Balík, Ph.D.}

\acknowledgements{I would like to thank my supervisor Ing. Miroslav Balík, Ph.D. for his good leadership and assistance. Big thanks go to my colleague Tomáš Zahálka for excellent cooperation. Thanks go to my family and friends for their support. Lastly, I would like to humbly thank myself for doing only a healthy level of procrastination.}

\abstractEN{The thesis goes through the entire process of developing a functional prototype of a mobile role-playing game with augmented reality features. The project is divided into two parts. First one is the Client which is installed at players' devices and allows them to interact with the game. The second one is the Server which provides a support for the Client, carries the game logic, and manages data persistence. This thesis deals only with development of the Server in Java language and involves challenges like designing and implementing an interface for clients to access the server features, creating a database structure capable of supporting a multiplayer game, and testing. The server features include support for interactions with game objects, such as fighting monster and using items, verification of in-app purchases made on Android system, and also geo-spatial search of game locations near a player. The prototype can be considered an Alpha version of the application and is ready to fully support the Client part.}

\abstractCS{Má bakalářská práce se zabývá procesem vývoje funkčního prototypu hry na hrdiny pro mobilní zařízení. Hra je určena pro více hráčů a obsahuje prvky rozšířené reality. Celý projekt je rozdělen na dvě části. První částí je Klient, aplikace nacházející se přímo na zařízeních hráčů a umožnující jim ovládat hru. Druhou částí je Server, který vykoná herní logiku a zajišťuje uchovávání dat. Tato práce se zabývá vývojem pouze serverové časti. Práce zahrnuje návrh a vytvoření rozhraní umožnující klientům pracovat s funkcemi serveru, dále návrh struktury databázových dat a také testování. Mezi funkce Serveru patří podpora akcí spojených s herními objekty, jako například zabíjení monster či používání objektů, ověřování mikrostransakcí provedených v systému Android a vyhledávání herních lokací v blízkosti hráče. Vzniklý prototyp odpovídá stavu Alfa verze aplikace a je připraven plně podporovat klientskou část.}

\placeForDeclarationOfAuthenticity{Prague}

\keywordsCS{herní server, Java, rozšířená realita, vyhledávání míst, funkční prototyp, mikrotransakce}

\keywordsEN{game server, Java, augmented reality, geo-spatial search, functional prototype, in-app purchases}

\declarationOfAuthenticityOption{1} %select as appropriate, according to the desired license (integer 1-6)


\begin{document}
	%\includepdf[pages={1}]{pdfs/official_assignment.pdf}
	%\newacronym{CVUT}{{\v C}VUT}{{\v C}esk{\' e} vysok{\' e} u{\v c}en{\' i} technick{\' e} v Praze}
	%\newacronym{FIT}{FIT}{Fakulta informa{\v c}n{\' i}ch technologi{\' i}}
	
	\setsecnumdepth{part}
	\chapter{Introduction}
	The world of mobile devices is quickly evolving. Smartphones and tablets are becoming more powerful and not only in terms of computational power and available memory. Nowadays, mobile devices are packed with various sensors. It is possible to acquire data from GPS (Global Positioning System) to quickly determine device’s position. This opens us door to augmented
reality (AR) applications. 

Surprisingly, there are not many existing augmented reality games on the market. Thanks to couple successful AR games in recent years, which I will explore later,  public awareness of this genre rapidly rose.

In my bachelor thesis, I will create the server part of a role-playing augmented reality game. I will work in cooperation with Tomáš Zahálka, who works on the client part. I will design and implement a prototype, test it and deploy it.
	
	
	\setsecnumdepth{all}
	\chapter{Project Overview}
	\section{Goals of the~Thesis}
The thesis is mainly focused on the~practical game server development. In the~research part, I will analyze existing augmented reality games, research available database management systems and explore frameworks commonly used for server development.

In the~practical part, I will specify features and requirements for the~server. The goal is to support actions the~mobile client might need to perform for seamless gaming experience. These operations include for example finding game objects near a~player, or getting user's profile information. The main goal is to implement server's functionality to communicate with clients, parse their requests and respond in valid format. The underlying goals include the~need to use a~database, caching, and to manage communication with external services. Lastly, the~game server prototype should be tested and deployed.	

\section{Game Description}
This is a~role-playing mobile game with augmented reality features. Player becomes a~character in an invisible alternative reality. He can explore the~new, magical world using his phone which displays a~map with objects around him. Player can discover monsters, like goblins or skeletons, and fight them to death for a~reward. As he gains more experience and gold, he can buy himself more powerful weapons and armor in a~shop. Tired of running in the~outside world? Player can exchange his real money for in-game gold.

For prototyping purposes, some described features are limited. All available actions are specified in \textit{Features} section of Requirements \ref{section:fr}.

\section{Game Story}
It was an ordinary Monday morning. Citizens of Flek, a~small city in the~heart of the~civilization, just woke up and poured to the~streets. Abruptly, a~loud cry pierced the~morning noise. 

“Make way! Make way!” yelled a~little boy while quickly elbowing his way through the~crowd, carefully holding a~piece of bread. People muttered silent curses, nevertheless they were moving out of his way to avoid his sharp elbows.

“Thief! Stop him! Stop the~boy!” suddenly echoed from the~other side of the~market square. City guards noticed the~boy had stolen the~bread from a~well-known baker. A hunt began, everyone was trying to take down the~boy. It didn’t take long and he was surrounded by the~guards, pinned down by a~local fat man who accidentally slipped and fell on the~boy. Guards tied him down, “Fiddle, again. We gave you a~warning the~last time. You are going with us.” City of Flek takes stealing very seriously. Poor Fiddle was sentenced to 15 years in prison.

Eight years later, a~boy became a~man. “Open up the~door,” someone said. The creaking door of the~cell slowly opened. A king himself entered. “Down on your knees, Fiddle.” said king. “I heard stories about you. I need someone like you, someone who has nothing to lose. I want you to go on a~quest. If~you succeed, you’ll be free.”

“What quest, Your Highness?” asked Fiddle.

“An old wizard told me, there are dragon eggs somewhere, still alive and ready to be hatched. As you might know, young dragons are easily trained to serve its owner. We are on a~verge of a~war with our neighbors. And they are stronger.” the~king sat down on Fiddle’s bed. “I want you to find those eggs and bring them to us. Dragons are extinct for over a~century. With dragon warriors, we’ll be the~most powerful nation in the~world.”

“I do not want to go on the~quest.” Fiddle looked up.

“Then you’ll be executed tomorrow.” the~king responded

“Ummm, alright then. You leave me no choice. I’ll find those dragon eggs for you.” said defeated Fiddle.

“Good. Take this device. It contains a~map of your surroundings. Now go!”

	
	\chapter{Analysis}
	
\section{Similar solutions}

	\subsection{Parallel Kingdom - Age of Ascension}
	This game was on market for 8 years (2008-2016). Parallel Kingdom is a closest solution to ours.
	
	
	”Parallel Kingdom is a mobile, location based, massively multiplayer game that uses GPS
	location and Google Maps to place users in a virtual world. Parallel Kingdom is the first
	location based RPG for the iOS and Android platforms. The game is set in a virtual world
	or ”Parallel Kingdom” where users claim their territories based on their GPS location or by making friends who invite them to travel to new places. Parallel Kingdom is a freemium
	game and utilizes a virtual goods revenue model.”
	
	\subsection{Ingress}
	Developed by Niantic, which was then part of Google, this game was released in 2013 for
	Android and in 2014 for iOS.[4] It is a location based, massively multiplayer game. A player have to choose one of the two factions, Enlightened or Resistance, and then as a part of his 	team capture regions of the game map. A faith of each faction relies on players’ cooperation. Thanks to that players meet in real life and coordinate their actions.
	
	Ingress was the first very successful augmented reality game with more than 10 000 000
	installs.
	
	\subsection{Pokémon GO}
	After its success with Ingress, Niantic started working on a new game Pokemon GO. Once
	released, the game became incredible hit. Even though the game faced many problems during
	its launch, mainly caused by the unexpected success and more active users than Pokémon
	GO was able to handle, in the first 80 days Pokémon GO reached about 550 million downloads and earned about \$470 million.
	
	The game is very similar to Ingress and uses the same crowd-sourced geographical data.
	
\section{Use Cases}

	\subsection{Use case diagram}
	
		\begin{figure}[h]	
			\includegraphics[width=\textwidth]{figures/UseCaseDiagram}
			\centering			
			\caption{Use case diagram}
			\label{fig:usecasediagram}
		\end{figure}
	
	\subsection{Use case descriptions}
		The following section refer to the use-cases introduced in Figure \ref{fig:usecasediagram}.
		\begin{enumerate}
			\item \textbf{Register account} \\
			\item \textbf{Login to the game} \\
			\item \textbf{Change the inventory content} \\
			\item \textbf{Go on a quest} \\
			\item \textbf{Fight a common monster} \\
			\item \textbf{Respawn} \\
			\item \textbf{Update a quest progress} \\
			\item \textbf{Create a new quest} \\
			\item \textbf{Edit or delete a quest} \\
						
		\end{enumerate}
	
	
\section{Requirements}

	\subsection{Functional}
		\begin{enumerate}
			\item \textbf{Users can use their Google accounts to login and play} \\
			For player's convenience, a Google account is required to play. The application does not have to store any password. 
			
			\item \textbf{Client can request nearby game locations} \\
			Upon client's request, the server has to respond with all game locations near the position of the client. The request can contain a filter.
			
			\item \textbf{Client can add, delete and update the player's inventory} \\
		\end{enumerate}
		
		
	\subsection{Non-functional}
	
		\begin{enumerate}
			\item \textbf{The data layer consists of a database engine and a caching} \\
			Game Server never accesses database directly. Data are retrieved from the database upon request and then cached. Upper layers communicate exclusively with cache.		
			
			\item \textbf{The server provides an API for client} \\
			The API support at least following:
			\begin{enumerate}
				\item Retrieve nearby game locations
				\item Create player
				\item Update player's data (inventory, experience, resources, quests progress etc.)
				\item Provide information about a quest 			
			\end{enumerate}
			
			\item \textbf{The communication between client and server parts of the application must be secure} \\
			All data sent from and to a client has to be encrypted.
	
			\item \textbf{Client can only connect to a Connection Server} \\
			Several Connection Servers exist to prevent a bottle-neck. Client selects the Connection Server by an algorithm. Client does not have an access to any other part of the server.
		\end{enumerate}
	\subsection{System and Interface}
	
		\begin{enumerate}
			\item \textbf{System uses Java 8 SE as an execution environment} \\
			
			\item \textbf{Operating system for the server is Debian 8} \\
			
			\item \textbf{Database engine is MySQL} \\
			
			\item \textbf{Cache engine is Redis} \\
		\end{enumerate}
	
\section{Technology}

	\subsection{Frameworks}
	
	
	\chapter{Design}
	\section{Activities}
	\subsection{Authentication}
	\begin{figure}[h]	
		\includegraphics[width=\textwidth]{figures/AD_Authentication}
		\centering			
		\caption{Activity diagram og the authentication process}
		\label{fig:adauth}
	\end{figure}

	\subsection{Retrieving nearby game objects}
	\begin{figure}[h]	
		\includegraphics[width=\textwidth]{figures/AD_Location}
		\centering			
		\caption{Activity diagram of how the server provides nearby game objects}
		\label{fig:adlocation}
	\end{figure}


\section{Database Model}
\begin{figure}[h]	
	\includegraphics[width=\textwidth, trim={0 5cm 0 0}]{figures/DatabaseModel}
	\centering
	\caption{Database model}
	\label{fig:dbmodel}	
\end{figure}
%
%	\subsection{Logical view}
%	
%	\subsection{Development view}
%	
%	\subsection{Process view}
%	
%	\subsection{Physical view}
%	
%	\subsection{Scenarios}
	
	\chapter{Implementation}
	\section{Development Environment}
I chose to use IntelliJ IDEA Ultimate 2017 \cite{idea} as my IDE. It is very user-friendly and powerful tool, which packs almost everything needed to develop a~Java application. Build management is handled by Apache Maven \cite{maven}. This tool is extremely useful as it takes care of all application's dependencies and completely manages the~build process. I also use GIT \cite{git} as my version control system. 

\section{Game Locations Source}
I have obtained all game locations from an open-source project OpenStreetMaps (OSM) \cite{osm}. I downloaded complete map data of the~Czech Republic. All map features in OSM have one or more tags which specify a~type of the~feature, for example \textit{amenity.college} is a~college or a~campus building, \textit{historic.castle} is a~castle and so on. Since it would be unreasonable to use all available features\footnote{The list of the~feature types is available at \url{http://wiki.openstreetmap.org/wiki/Map_Features}}, I chose only several types, mostly from categories \textit{amenity} and \textit{historic}. I used a~tool \textit{Osmosis} \cite{osmosis} to extract selected map features from the~data. The selection resulted in 99~037 locations for the~entire Czech Republic.

\section{Database}
The initial database structure was created from the~database model shown in Figure \ref{fig:dbmodel}. I used MySQL Workbench \cite{mysqlworkbench} to generate a~creation script.

The database includes an event which is triggered every 8 hours and wipes a~table containing list of monsters killed by users. It causes the~monsters to re-spawn.  

\section{Project Structure}
The entire project is organized by its components (also called \textit{modules} in the~IDEA's terminology). The project name is \textit{BachelorsServer} and consists of three modules -- \textit{ConnectionServer}, \textit{LoginServer}, and \textit{DatabaseServer}. The project source code is located in folder \textit{/server/BachelorsServer/} on the~enclosed SD card. Each module follows Maven's Standard Directory Layout\footnote{Specification the~Standard Directory Layout is available at \url{https://maven.apache.org/guides/introduction/introduction-to-the-standard-directory-layout.html}}. Top-level directory contains important configuration file. First one is \textit{config.yml} which stores server setting, such as listening ports, used protocol (HTTP/HTTPS), or URLs of other components. Path to this file must be specified as command-line argument of the~server. Second file is \textit{pom.xml} (\textit{Project Object Model} file). It contains project-specific definitions for Maven. It specifies project version and name, its dependencies, and build strategies. 

\noindent Common package organization is\footnote{The package name \textit{module} is used as a~placeholder for module-specific name -- database, connection, or login.}:
\begin{description}
	\item[bachelors.\textit{module}] Main class and server configuration classes.
	\item[bachelors.\textit{module}.api] Classes used for JSON serialization and deserialization.
	\item[bachelors.\textit{module}.resources] Definitions of API endpoints.
\end{description}

\section{Connection Server}
A component which mostly serves as a~proxy. Connection server is designed to be lightweight, since every client connects through this component. Connection Server is the~least complex module of the~three. To satisfy the~requirement for secured communication, this component can be configured to allow incoming connections only via HTTPS protocol.  

\subsection{Resources}
Classes which handle API requests. I described their main responsibilities and examples of their usage in the~following text.

\begin{description}	
	\item[BaseResource] Abstract super-class for all resources. It contains commonly used methods, such as \textit{putRequest()}, or \textit{getRequest()}. These methods verify client's access code and delegates the~request to a~supplied URL.
	
	\item[UserResource] The class handles user-oriented requests. It is used to get user's profile, or his inventory.
	
	\item[LoginResource] The resource oriented on login and registration requests.
	
	\item[LocationResource] It is responsible for retrieving nearby game locations.
	
	\item[PurchaseResource] Contains endpoints which handle in-app purchases.

	\item[ActionResource] Important resource which processes user's action. It is used to kill a~monster, to buy an item from a~shop, or to use a~health potion.
\end{description}

\section{Login Server}
A component responsible for authentication and authorization of clients and in-app purchase verification. I introduced several dependencies to help me fulfill the~requirements.
 
\subsection{Access Key Store}
All client's requests after login are authorized using an access code. The storage for these codes has to be fast, reliable and shared among all instances of Login server component. The codes are stored in Redis and accessed from the~component using a~Redis java client -- Jedis \cite{jedis}. 

\subsection{Google API}
During the~authentication process, a~user sends a~Google ID token. The component uses a~Google API Client library \cite{googleapilibs} to access Google API and exchange the~token for user ID and e-mail. The in-app purchase verification follows similar process. The task is done using Google Play Developer API Client Library for Java  \cite{androidpublisherlibary}.

\section{Database Server}
The most complex and important component of the~three. It is responsible for game logic and data persistence.

\subsection{Configuration Files}
A folder \textit{BachelorsServer/DatabaseServer/src/main/resources} contains configuration files for various dependencies of Database Server.
\begin{description}
	\item[hibernate.cfg.xml] Hibernate's running configuration. It defines second-level cache settings, type of the~database engine, and lists database entities.
	\item[hibernate-redis.properties] The file sets up Hibernate's second-level cache provider.
	\item[logback.xml] Settings of logger. Specifies what log level to show and where to output the~logs.
	\item[redisson.yml] Configuration for Redisson which describes Redis server connection settings. 
\end{description}

\subsection{Hibernate}
The component uses Hibernate library to access the~database.

\subsubsection*{Entities}
Every table in the~database (except Many-to-Many relationships) has to be defined in an entity class. The specification contains not only table columns but also other entities in relationship with the~class. It means the~developer can for example simply call \textit{getGameObjectType()} on \textit{GameObject} and Hibernate automatically fetches associated \textit{GameObjectType} from the~database.

\subsubsection*{Second-level cache}
I've decided to use a~second-level cache to optimize database interactions. Hibernate supports many cache providers and I chose to use hibernate-redis \cite{hibernateredis}. When the~cache is configured, each entity can be annotated as \textit{@Cacheable} and have caching strategy specified. Hibernate then automatically caches the~annotated entities.

\subsection{Models}
Package \textit{bachelors.database.db} contains models in which database operations are implemented. Each model extends abstract super-class \textit{BaseModel} and operates in its logical scope, e.g. \textit{UserModel} handles user-related operations, \textit{GameObjectModel} handles game object related actions and so on. The \textit{BaseModel} implements generic methods for simple database operations, like select all objects, or get object by id. Please refer to class diagram in Figure \ref{fig:models} or to project documentation for more detailed information about models.

\subsection{HTTP Basic Authentication} 
Access to \textit{Admin} endpoints is allowed only to authorized users. Database Server uses Dropwizard's authentication support. Only three additional files are needed to implement HTTP Basic Authentication. They are all located in package \textit{bachelors.database.security}. 
\begin{description}
	\item[AdminUser] Type of the~user which extends \textit{java.security.Principal}.
	\item[AdminAuthenticator] The class defines how the~application verifies username and password.
	\item[AdminAuthorizer] A method to verify if the~user has sufficient privileges to access the~requested resource.
\end{description}

\section{Deployment Environment}
\label{section:deploy}
The prototype currently runs on a~virtual private server\footnote{The server is accessible at https://31.31.78.223:8080 [as of 2017/06/20]} provided by WEDOS internet, a.s. The server's operating system is Debian~8. It uses the~lowest available server specification:
\begin{tabbing}
	\textit{CPU} ~~ \= 1 thread Xeon 1.70 GHz\\
	\textit{RAM} \> 2 GB\\
	\textit{SSD} \> 15 GB\\
	\textit{SLA} \> 99.99\%
\end{tabbing}

The specifications are sufficient for testing and prototyping but in unused state, the~server has only about 200~MB free RAM. Additionally, the~application will have to support many concurrent users which requires more CPU threads.  

\section{Documentation}
The source code of the~application includes a~documentation in JavaDoc notation. Generated documentation is available as interactive HTML documents on the~enclosed SD card at \textit{/documentation}.
	
	\chapter{Testing}
	\section{Unit Testing}
The project use JUnit \cite{junit}. Unit tests are located in \textit{src/test} directory. All components use the test to verify proper serialization and deserialization of JSON requests. Each tested JSON object is defined in \textit{src/test/resources/fixtures}.

The tests are packed in a test suite. This allows running all test by executing \textit{bachelors.connection.api.AllTests}. All tests are passing in the current build.

\section{Static Code Analysis}
I use SonarLint \cite{sonarlint} for on-the fly static code analysis. It offers many useful rules with specified severity and also a description how to fix the issues. Static code analysis discovered many issues, the most notable were:
\begin{itemize}
	\item Hide a private constructor to hide the implicit one in a static class.
	\item Use constant instead of duplicating string literal.
	\item Replace use of System.out by a logger.
	\item Make final constants also static.
\end{itemize}
The static analysis proved to be useful to maintain a good quality of the code and prevent bugs.

\section{System Testing}
I developed a Python script to test most of the endpoints in the real environment. It skips tests of \textit{/login/registration} and \textit{/purchase} as they need valid Google tokens which can't be reused and are not easily obtainable. All tests are passing in the current build.

The script requires Python 3 or newer to run and is located in \textit{BachelorsServer/SystemTest/TestBachelorsPrototype.py}. The tester must initialize the script with values of a tests account in the current database. The script then sends HTTP request to each endpoint testing valid and invalid data.

For example, test \textit{ACTION -- Kill} verifies the client cannot request setting his health to negative value or more than his level allows. Script also tries to kill a monster at a wrong location and vice versa. A successful kill follows a second attempt to kill the same monster which must fail. 


\section{Client Testing}
I tested the application with my colleague using his client-part of the game. We didn't discover any bugs which would affect the player in any way.

\section{Stress Testing}
I performed a stress test using Apache JMeter \cite{jmeter}. This testing framework is used to simulate interactions with a web server. Results of this type of test offer useful insight into how many concurrent users can server handle. Hardware specifications of the testing environment are described in Section \ref{section:deploy}.

I created a test plan in a way, that one thread simulates about 10 users. Each thread operates independently for 10 minutes and performs following series of actions:
\begin{enumerate}
	\item Wait 1 second\footnote{The locations are fetched from the server once every 10 seconds in real game client}
	\item Get nearby locations around location A (\textit{/location})
	\item Wait 1 second
	\item Get nearby locations around location B (\textit{/location})
	\item Wait 1 second
	\item Get nearby locations around location C (\textit{/location})
	\item Get profile (\textit{/user})
	\item Get inventory (\textit{/user/inventory})
	\item Go back to step 1
\end{enumerate}

As expected, calls to the endpoint \textit{/location} took the longest time out of the three tested, the difference in latency was about 30\% on average for 500 and 1~500 users. As you can see in Table \ref{tab:loadtestresults}, latency significantly increased at 3~000 users and about 2\% request resulted in an error. At 5 000 users, the server was far beyond its limits and responded with very high latency and about 8\% error rate. This behavior is expected due to the very limited resources of the server. The results might be affected by the performance issues of the test machine, since 500 threads had to be run concurrently to simulate 5 000 users.

Based on the test, server is currently able to handle more than 2~300 concurrent users with no performance issues that would affect clients.

\begin{table}
	\centering
	\begin{tabular}{r || c | c | c | c | c |}
		Users & 500 & 1 500 & 2 300 & 3 000 & 5 000 \\ \hline 
		Latency median [ms] & 108 & 123 & 166 & 473 & 1 737 \\
		Throughput [request/s] & 16.0 & 46.4 & 67.9 & 79.1 & 99.8 \\
	\end{tabular}
	\caption{Result of the server stress test. \cite{dbengines}}
	\label{tab:loadtestresults}
\end{table}	


	
	\setsecnumdepth{part}
	\chapter{Conclusion}
	This thesis aimed to create a prototype of the server part of a role-playing game with features of augmented reality. I explored and analyzed exiting similar games. After discussion with my colleague Tomáš Zahálka, I defined the rules and features of the game prototype. 

I analyzed use cases and specified requirements to clarify expected server functionality. I explored available solutions for \textit{database management system} and chose t o use \textit{MySQL}. I decided to use several Java frameworks and libraries to handle database communication, JSON serialization, and manage API endpoints. I designed the structure of the server and described actions a user can perform. Based on these actions, I created a specification for public API as well as private one for internal communication among components. I designed and implemented a database model. I obtained and imported initial game locations to the database. In implementation phase, I created all specified API endpoints, implemented game logic a database communication. Administration section was secured and requires authentication. I created an index of location and configured a cache to improve database performance. Lastly I performed unit, system  and stress tests as well as static code analysis.

Since the server is currently in prototype version, many features and further improvements are planed for future development. I plan to improve security, error handling and increase coverage of the unit tests before production release. One of the planned features is a quest system. My design allow easy scalability to support many concurrent users and high extensibility which enables me to create full, market-ready server.
	
	\printbibliography
	
	\setsecnumdepth{all}
	\appendix
	
	\chapter{Acronyms}
	% \printglossaries
	\begin{description}
		\item[API] Application Programming Interface
		\item[CPU] Central Processing Unit
		\item[CS] Connection Server
		\item[DBMS] Database Management System
		\item[DS] Database Server
		\item[HTTP] Hypertext Transfer Protocol
		\item[HTTPS] Hypertext Transfer Protocol Secure	
		\item[ID] Identifier
		\item[IDE] Integrated Development Environment
		\item[JSON] JavaScript Object Notation
		\item[LS] Login Server
		\item[RAM] Random-Access Memory
		\item[REST] Representational State Transfer
		\item[SLA] Service Level Agreement	
		\item[UID] Unique Identifier
		\item[URI] Uniform Resource Identifier
		\item[URL] Uniform Resource Locator		
	\end{description}
	
	\chapter{API}
	\label{appendix:api}
	\section{Public API}
Following text describes public API available to clients. All error responses return \textbf{Error} (\ref{json:error}) with a description of the error.

	\subsection{GET /login}
	The Login endpoint verifies a Google ID \textit{token} and generates an \textit{access code} for future request. When successfully authenticated, user's profile and the \textit{access code} is returned.
		\paragraph*{Parameters}
			\begin{description}
				\item[token] Google ID token [string]
			\end{description}
		\paragraph*{Responses}
			\begin{description}
				\item[200] User successfully logged in, return \textbf{Profile} (\ref{json:profile}) with \textit{access code} set.
				\item[403] Invalid \textit{token}
				\item[404] User not found
				\item[500] Unexpected error
			\end{description}
	
	\subsection{POST /login/register}
	The Registration endpoint creates a new user on the server. His profile is initialized with default values. If the \textit{username} is taken or if the user already exists, then an error is returned.
	
		\paragraph*{Body}
			\begin{description}
				\item[UsernameWToken] Defined in \ref{json:usernamewtoken}.
			\end{description}
		\paragraph*{Responses}
			\begin{description}
				\item[200] User successfully registered and logged in, return \textbf{Profile} (\ref{json:profile}) with \textit{access code} set.
				\item[403] Invalid \textit{token}
				\item[409] Either \textit{username} exists or the user is already registered.
				\item[500] Unexpected error
			\end{description}	

	\subsection{GET /user}
	The User endpoint returns the user profile.
		\paragraph*{Parameters}
			\begin{description}
				\item[accessCode] \textit{Access code} [string]
			\end{description}
		\paragraph*{Responses}
			\begin{description}
				\item[200] Return \textbf{Profile} (\ref{json:profile}).
				\item[403] Invalid \textit{access code}
				\item[404] User not found
				\item[500] Unexpected error
			\end{description}

	
	\subsection{PUT /user/die}	
	The Die endpoint kills a user. He's punished with a \textit{gold} penalty and his \textit{health} is restored.
		\paragraph*{Body}
			\begin{description}
				\item[AccessCode] Defined in \ref{json:accesscode}.
			\end{description}
		\paragraph*{Responses}
			\begin{description}
				\item[200] Return \textbf{Profile} (\ref{json:profile}).
				\item[403] Invalid \textit{access code}
				\item[404] User not found
				\item[500] Unexpected error
			\end{description}
	
	\subsection{GET /user/inventory}
	The User Inventory endpoint return all the items in the user's inventory and the information about what is equipped in which slot.
		\paragraph*{Parameters}
			\begin{description}
				\item[accessCode] \textit{Access code} [string]
			\end{description}
		\paragraph*{Responses}
			\begin{description}
				\item[200] Return \textbf{InventoryWEquipment} (\ref{json:inventoryequipment}).
				\item[403] Invalid \textit{access code}
				\item[404] User not found
				\item[500] Unexpected error
			\end{description}
		
	\subsection{GET /location}
	The Location endpoint retrieves all nearby locations in radius 200 m from the provided coordinates. The locations are returned along with their associated game objects.
		\paragraph*{Parameters}
			\begin{description}
				\item[accessCode] \textit{Access code} [string]
				\item[lat] Latitude [double]
				\item[lon] Longitude [double]
			\end{description}
		\paragraph*{Responses}
			\begin{description}
				\item[200] Return list of nearby \textbf{Location} (\ref{json:location}) along with their assigned \textit{game object}.
				\item[403] Invalid \textit{access code}
				\item[404] User not found
				\item[500] Unexpected error
			\end{description}
	
	\subsection{POST /action/kill}
	The Kill endpoint performs kill action on the selected object and location. The location will be temporarily excluded from future requests to \textit{/location}. User \textit{health} will be updated, \textit{experience} and \textit{gold} will be added. The endpoint returns \textit{killConfirmedCode} which is needed to perform collect action
		\paragraph*{Body}
			\begin{description}
				\item[Kill] Defined in \ref{json:kill}.
			\end{description}
		\paragraph*{Responses}
			\begin{description}
				\item[200] Return \textbf{KillConfirmedCode} (\ref{json:killconfirm}).
				\item[400] Invalid data
				\item[403] Invalid \textit{access code}
				\item[404] User not found
				\item[500] Unexpected error
			\end{description}
	
	\subsection{POST /action/collect}
	The Collect endpoint allows collecting items from \textit{monster's} inventory after kill. It can be called only once after each kill.
		\paragraph*{Body}
			\begin{description}
				\item[Collect] Defined in \ref{json:collect}.
			\end{description}
		\paragraph*{Responses}
			\begin{description}		
				\item[200] Return \textbf{InventoryWEquipment} (\ref{json:inventoryequipment}).
				\item[400] Invalid data
				\item[403] Invalid \textit{access code}
				\item[404] User not found
				\item[500] Unexpected error
			\end{description}
	
	\subsection{PUT /action/equip}
	The Equip endpoint equips an item from user's inventory to the specified slot.
		\paragraph*{Body}
			\begin{description}
				\item[Equip] Defined in \ref{json:equipment}.
			\end{description}
		\paragraph*{Responses}
			\begin{description}		
				\item[200] Successfully equipped item
				\item[400] Invalid data
				\item[403] Invalid \textit{access code}
				\item[404] User not found
				\item[500] Unexpected error
			\end{description}
	
	\subsection{POST /action/use}
	The Use endpoint uses an item from user's inventory and adds \textit{health}, \textit{gold}, or \textit{experience}.
		\paragraph*{Body}
			\begin{description}
				\item[Use] Defined in \ref{json:use}.
			\end{description}
		\paragraph*{Responses}
			\begin{description}		
				\item[200] Return \textbf{Profile} (\ref{json:profile}).
				\item[400] Invalid data
				\item[403] Invalid \textit{access code}
				\item[404] User not found
				\item[500] Unexpected error
			\end{description}
	
	\subsection{POST /action/buy}
	The Buy endpoint buys an item from a \textit{shop}.
		\paragraph*{Body}
			\begin{description}
				\item[Buy] Defined in \ref{json:buy}.
			\end{description}
		\paragraph*{Responses}
			\begin{description}		
				\item[200] Return \textbf{InventoryWEquipment} (\ref{json:inventoryequipment}).
				\item[400] Invalid data
				\item[403] Invalid \textit{access code}
				\item[404] User not found
				\item[500] Unexpected error
			\end{description}
	
	\subsection{POST /purchase}
	The Purchase endpoint processes in-app purchases on server. The purchase must be valid and not consumed to be accepted.
		\paragraph*{Body}
			\begin{description}
				\item[Purchase] Defined in \ref{json:purchase}.
			\end{description}
		\paragraph*{Responses}
			\begin{description}		
				\item[200] Return \textbf{Profile} (\ref{json:profile}).
				\item[400] Invalid data
				\item[403] Invalid \textit{access code}
				\item[404] User not found
				\item[500] Unexpected error
			\end{description}

\section{Admin API}
The following endpoints are protected with HTTP Basic Authentication. Authorized users can use them to manage game objects, locations, and game object types.

	\subsection{PUT /admin/reindex}
	The Reindex endpoint clears the location index and re-initializes it.
		\paragraph*{Responses}
		\begin{description}		
			\item[200] Successfully reindexed
			\item[500] Unexpected error
		\end{description}

	\subsection{POST /admin/importLocations}
	The Import locations endpoint parses an XML file with OpenStreetMap locations and imports them into the database.
		\paragraph*{Body}
			\begin{description}
				\item[ImportLocations] Defined in \ref{json:importlocations}.
			\end{description}
		\paragraph*{Responses}
			\begin{description}		
				\item[200] Return number of imported locations
				\item[500] Unexpected error
			\end{description}
		
	\subsection{DELETE /admin/clearCache}
	The Clear cache locations endpoint parses an XML file with OpenStreetMap locations and imports them into the database.
		\paragraph*{Responses}
			\begin{description}		
				\item[200] Cache successfully cleared
				\item[500] Unexpected error
			\end{description}
			
	\subsection{GET /admin/gameObject}
	The GET Game object endpoint returns all available game objects.
		\paragraph*{Responses}
			\begin{description}		
				\item[200] List of \textbf{GameObject} (\ref{json:gameobject})
				\item[500] Unexpected error
			\end{description}
			
	\subsection{POST /admin/gameObject}
	The POST Game object endpoint creates a new game object of a specified type.
		\paragraph*{Responses}
			\paragraph*{Body}
				\begin{description}
					\item[AdminGameObject] Defined in \ref{json:admingameobject}. Field \textit{id} not set.
				\end{description}
			\begin{description}		
				\item[200] Game object successfully created
				\item[400] Invalid data
				\item[500] Unexpected error
			\end{description}
			
	\subsection{PUT /admin/gameObject}
	The PUT Game object endpoint replaces the game object's children with the ones provided in the request.
		\paragraph*{Responses}
			\paragraph*{Body}
				\begin{description}
					\item[AdminGameObject] Defined in \ref{json:admingameobject}. Fields \textit{id} and \textit{childrenIds} are mandatory.
				\end{description}
			\begin{description}		
				\item[200] Game object's children successfully updated
				\item[400] Invalid data
				\item[500] Unexpected error
			\end{description}
			
	\subsection{GET /admin/gameObjectType}
	The GET Game object type endpoint returns all available game object types.
		\paragraph*{Responses}
			\begin{description}		
				\item[200] Return all game object types
				\item[400] Invalid data
				\item[500] Unexpected error
			\end{description}
			
	\subsection{POST /admin/gameObjectType}
	The POST Game object type endpoint creates a new game object type with specified actions and attributes. 
		\paragraph*{Body}
			\begin{description}
				\item[AdminGameObjectType] Defined in \ref{json:admingameobjecttype}.
			\end{description}
		\paragraph*{Responses}
			\begin{description}		
				\item[200] Game object type successfully created
				\item[400] Invalid data
				\item[500] Unexpected error
			\end{description}
			
	\subsection{PUT /admin/gameObjectType}
	The PUT Game object type endpoint updates the game object type.
		\paragraph*{Body}
			\begin{description}
				\item[AdminGameObjectType] Defined in \ref{json:admingameobjecttype}. Field \textit{id} is mandatory.
			\end{description}
		\paragraph*{Responses}
			\begin{description}		
				\item[200] Return all game object types
				\item[400] Invalid data
				\item[500] Unexpected error
			\end{description}			
				

\section{JSON Objects}
Definitions of objects sent in HTTP response or request body. The objects are of the type \textit{application/json}.
	\subsection{Profile}
		\label{json:profile}
		Complete user's profile information. May include \textit{access code}.
		\subsubsection{Schema}	
			\begin{description}
				\item[id] User ID [string]
				\item[username] Username [string]
				\item[email] E-mail [string]
				\item[active] Account activity status [boolean]
				\item[health] Current health [integer]
				\item[experience] Total experience [integer]
				\item[gold] Owned gold [integer]
				\item[gems] Owned gems [integer]
				\item[accessCode] Access code [integer]
			\end{description}

	\subsection{UsernameWToken}
		\label{json:usernamewtoken}
		Wrapper entity for registration data.
		\subsubsection{Schema}
			\begin{description}
				\item[username] Username [string]
				\item[token] Google ID token [string]
			\end{description}

	\subsection{Error}
		\label{json:error}
		Entity returned when an error occurs. It describes what happened.
		\subsubsection{Schema}
			\begin{description}
				\item[code] Error code [integer]
				\item[reason] Explanation of the error [string]
			\end{description}

	\subsection{AccessCode}
		\label{json:accesscode}
		Wrapper entity for \textit{access code}.
		\subsubsection{Schema}
			\begin{description}
				\item[accessCode] Access code
			\end{description}

	\subsection{InventoryWEquipment}
		\label{json:inventoryequipment}
		Wrapper entity for user's inventory and his equipment.		
		\subsubsection{Schema}
			\begin{description}
				\item[inventory] List of \textbf{InventoryObject} (\ref{json:inventoryobject}) [array of objects]
				\item[equipment] User's \textbf{Equipment} (\ref{json:equipment}) [object]
			\end{description}
	
	\subsection{Equipment}
		\label{json:equipment}
		Entity which contains all slots a player can equip items to. Each slot contains ID of the equipped item or null, if empty. 
		\subsubsection{Schema}
			\begin{description}
				\item[feet] Item in feet slot [integer]
				\item[legs] Item in legs slot [integer]
				\item[chest] Item in chest slot [integer]
				\item[head] Item in head slot [integer]
				\item[necklace] Item in necklace slot [integer]
				\item[belt] Item in belt slot [integer]
				\item[leftHand] Item in leftHand slot [integer]
				\item[rightHand] Item in rightHand slot [integer]
				\item[dualHand] Item in dualHand slot [integer]
				\item[item] Item in item slot [integer]			
			\end{description}

	\subsection{InventoryObject}
		\label{json:inventoryobject}	
		Item in user's inventory.
		\subsubsection{Schema}
			\begin{description}
				\item[id] Item ID [integer]
				\item[gameObjectTypeId] ID of the type of the item.	[integer]		
			\end{description}

	\subsection{Location}
		\label{json:location}	
		Description of a game location. Includes a \textit{game object} assigned to it.
		\subsubsection{Schema}
			\begin{description}
				\item[id] Location ID [integer]
				\item[latitude] Latitude [double]
				\item[latitude] Longitude [double]
				\item[gameObject] \textbf{GameObject} (\ref{json:gameobject}) assigned to the location [object]
			\end{description}	

	\subsection{GameObject}
		\label{json:gameobject}
		Game object like \textit{monster} or \textit{shop}. Includes its inventory.

		\subsubsection{Schema}
			\begin{description}
				\item[id] Game object ID [integer]
				\item[name] Custom name [string]
				\item[description] Custom description [string]
				\item[gameObjectTypeId] ID of the type of the game object [integer]
				\item[gameObjects] List of \textbf{GameObject} (\ref{json:gameobject}) contained in this one [array of objects]
			\end{description}
	
	\subsection{Kill}
		\label{json:kill}
		Information about the user's kill. Describes the kill result.
		\subsubsection{Schema}
			\begin{description}
				\item[accessCode] Access code [string]
				\item[locationId] ID of the location where the kill happened [integer]
				\item[gameObjectId] ID of the killed monster [integer]
				\item[health] User's health after kill [integer]
			\end{description}
	
	\subsection{KillConfirmedCode}
		\label{json:killconfirm}
		Wrapper entity for kill confirmation.
		\subsubsection{Schema}
			\begin{description}
				\item[killConfirmedCode] Confirmation code [integer]
			\end{description}
	
	\subsection{Collect}
		\label{json:collect}
		Specification of what user wants to collect from his killed \textit{monster}.
		\subsubsection{Schema}
			\begin{description}
				\item[accessCode] Access code [string]
				\item[killConfirmedCode] Confirmation code for the kill [integer]
				\item[gameObjects] List of IDs of the game objects user wants to collect [array of integers]
			\end{description}
	
	\subsection{Equip}
		\label{json:eqip}
		Entity describing what item from user's inventory to equip to which slot.
		\subsubsection{Schema}
			\begin{description}
				\item[accessCode] Access code [string]
				\item[itemId] ID of the item from user's inventory to equip [integer]
				\item[slot] Name of the slot in which to equip the item [string]
			\end{description}

	\subsection{Use}
		\label{json:use}
		Describes what item from user's inventory to use.

		\subsubsection{Schema}
			\begin{description}
				\item[accessCode] Access code [string]
				\item[itemId] ID of the item from user's inventory to use [integer]
			\end{description}
	
	\subsection{Buy}
		\label{json:buy}
		Entity which specifies what item user wants to buy and from what \textit{shop}.

		\subsubsection{Schema}
			\begin{description}
				\item[accessCode] Access code [string]
				\item[shopId] ID of the shop from which to buy [integer]
				\item[itemId] ID of the item to buy [integer]
			\end{description}
	
	\subsection{Purchase}
		\label{json:purchase}
		Entity which bears information about an in-app purchase.
		\subsubsection{Schema}
			\begin{description}
				\item[accessCode] Access code [string]
				\item[productId] ID of the purchased product [string]
				\item[token] Purchase token from Google [string]
			\end{description}
		
	\subsection{ImportLocations}
		\label{json:importlocations}
		Admin entity for importing locations from an XML file.
		\subsubsection{Schema}
			\begin{description}
				\item[tag] OpenStreetMap category tag [string]
				\item[source] Path to the XML file on server [string]
			\end{description}
		
	\subsection{AdminGameObject}
		\label{json:admingameobject}
		Admin entity describing a game object.
		\subsubsection{Schema}
			\begin{description}
				\item[id] ID of an existing game object [integer, optional]
				\item[gameObjectTypeId] ID of the type of the game object [integer]
				\item[root] True if the object cannot have ancestors, false otherwise [boolean]
				\item[childrenIds] IDs of the game object's children [array of integers]
			\end{description}
			
	\subsection{AdminGameObjectType}
		\label{json:admingameobjecttype}
		Admin entity describing a game object type.
		\subsubsection{Schema}
			\begin{description}
				\item[id] ID of an existing game object type [integer, optional]
				\item[name] Name [string]
				\item[description] Description [string]
				\item[attributes] Game object type attributes [array of objects]
				\item[actions] All alowed actions[array of strings]
			\end{description}
		
	
	
	
	\chapter{Class Diagrams}
	\label{appendix:classdiagrams}
	\section{Connection Server}
\begin{figure}[h]	
	\includegraphics[width=\textwidth]{figures/classdiagrams/csconnection}
	\centering			
	\caption{Class diagram of package \textit{bachelors.connection}}
\end{figure}

\begin{figure}[h]	
	\includegraphics[width=\textwidth]{figures/classdiagrams/csresources}
	\centering			
	\caption{Class diagram of package \textit{bachelors.connection.resources}}
\end{figure}

\begin{figure}[h]	
	\includegraphics[width=\textwidth]{figures/classdiagrams/csapi}
	\centering			
	\caption{Class diagram of package \textit{bachelors.connection.api}}
\end{figure}

\begin{figure}[h]	
	\includegraphics[height=0.9\textheight]{figures/classdiagrams/csaction}
	\centering			
	\caption{Class diagram of package \textit{bachelors.connection.api.action}}
\end{figure}




\section{Database Server}
\begin{figure}[h]	
	\includegraphics[width=\textwidth]{figures/classdiagrams/dsdatabase}
	\centering			
	\caption{Class diagram of package \textit{bachelors.database}}
\end{figure}

\begin{figure}[h]	
	\includegraphics[width=\textwidth]{figures/classdiagrams/dsresources}
	\centering			
	\caption{Class diagram of package \textit{bachelors.database.resources}}
\end{figure}

\begin{figure}[h]	
	\includegraphics[width=\textwidth]{figures/classdiagrams/dsapi}
	\centering			
	\caption{Class diagram of package \textit{bachelors.database.api}}
\end{figure}

\begin{figure}[h]	
	\includegraphics[width=\textwidth]{figures/classdiagrams/dsaction}
	\centering			
	\caption{Class diagram of package \textit{bachelors.database.api.action}}
\end{figure}

\begin{figure}[h]	
	\includegraphics[width=\textwidth]{figures/classdiagrams/dsadmin}
	\centering			
	\caption{Class diagram of package \textit{bachelors.database.admin}}
\end{figure}

\begin{figure}[h]	
	\includegraphics[width=0.6\textwidth]{figures/classdiagrams/dscore}
	\centering			
	\caption{Class diagram of package \textit{bachelors.database.core}}
\end{figure}

\begin{figure}[h]	
	\includegraphics[width=\textwidth]{figures/classdiagrams/dsexception}
	\centering			
	\caption{Class diagram of package \textit{bachelors.database.exception}}
\end{figure}

\begin{figure}[h]	
	\includegraphics[width=\textwidth]{figures/classdiagrams/dstype}
	\centering			
	\caption{Class diagram of package \textit{bachelors.database.type}}
\end{figure}

\begin{figure}[h]	
	\includegraphics[width=\textwidth]{figures/classdiagrams/dssecurity}
	\centering			
	\caption{Class diagram of package \textit{bachelors.database.security}}
\end{figure}

\begin{figure}[h]	
	\includegraphics[width=\textwidth]{figures/classdiagrams/dsdb}
	\centering			
	\caption{Class diagram of package \textit{bachelors.database.db}}
	\label{fig:models}
\end{figure}

\begin{figure}[h]	
	\includegraphics[width=\textwidth]{figures/classdiagrams/dsentity1}
	\centering			
	\caption{Part 1/3 of a class diagram of package \textit{bachelors.database.entity}}
\end{figure}

\begin{figure}[h]	
	\includegraphics[width=\textwidth]{figures/classdiagrams/dsentity2}
	\centering			
	\caption{Part 2/3 of a class diagram of package \textit{bachelors.database.entity}}
\end{figure}

\begin{figure}[h]	
	\includegraphics[width=\textwidth]{figures/classdiagrams/dsentity3}
	\centering			
	\caption{Part 3/3 of a class diagram of package \textit{bachelors.database.entity}}
\end{figure}




\section{Login Server}

\begin{figure}[h]	
	\includegraphics[width=\textwidth]{figures/classdiagrams/lslogin}
	\centering			
	\caption{Class diagram of package \textit{bachelors.login}}
\end{figure}

\begin{figure}[h]	
	\includegraphics[width=\textwidth]{figures/classdiagrams/lsresources}
	\centering			
	\caption{Class diagram of package \textit{bachelors.login.resources}}
\end{figure}

\begin{figure}[h]	
	\includegraphics[width=\textwidth]{figures/classdiagrams/lsapi}
	\centering			
	\caption{Class diagram of package \textit{bachelors.login.api}}
\end{figure}

\begin{figure}[h]	
	\includegraphics[width=\textwidth]{figures/classdiagrams/lsgoogle}
	\centering			
	\caption{Class diagram of package \textit{bachelors.login.google}}
\end{figure}

\begin{figure}[h]	
	\includegraphics[width=\textwidth]{figures/classdiagrams/lssecurity}
	\centering			
	\caption{Class diagram of package \textit{bachelors.login.security}}
\end{figure}
	
	\chapter{Installation Instructions}
	\section{Prerequisites}
The server has several dependencies which should be met before the installation process. Even though the target platform is Debian, the server should work on any operating system with Java support. While the server might work with software versions other than the ones I included, the compatibility is not guaranteed. The dependencies can be downloaded for free from the cited websites.

\begin{enumerate}
	\item MySQL 5.7 \cite{mysql}
	\item Redis 3.2 \cite{redis}
	\item Java 8 or OpenJDK 8	
\end{enumerate}

\section{Compilation}
Although I enclosed compiled \textit{JAR} files on the SD card, the source code can be compiled using Maven \cite{maven}. Simply move to the root directory of a component and execute a Maven goal \textit{install} and optionally a goal \textit{clean}:\\
\emph{maven clean install}

Compiled \textit{JAR} binaries and generated HTML documentation can be found in \textit{target/} folder upon successful compilation.

\section{Database Initialization}
The database must be initialized prior running the Database server component. I've created an SQL script which creates the database structure and imports initial data, such as game objects, locations, actions, and so on. The script is located at \textit{/BachelorsServer/Data/init\_database.sql} on the enclosed SD card.

The MySQL server might not include timezone information rendering the application broken. You can import the time zones converting system time zones to SQL using\textit{mysql\_tzinfo\_to\_sql} tool and piping the output to \textit{mysql} program \cite{mysqltimezones}:\\
\emph{mysql\_tzinfo\_to\_sql /usr/share/zoneinfo | mysql -u root mysql}

\section{Configuration}
Default server configuration is available in \textit{config.yml} file which can be found in the root folder of each component on the enclosed SD card. The content of the configuration file is in YAML\footnote{YAML format specification can be found at \url{http://www.yaml.org/spec/1.2/spec.html}} format.The following text describes the meaning of important parameters.

\paragraph*{Connection Server}
\begin{description}
	\item[loginServerUrl] URL of a Login Server instance (example: \textit{http://localhost:8090})
	\item[databaseServerUrl] URL of a Database Server instance (example: \textit{http://localhost:8092})
	\item[server] Jetty web server configuration
		\begin{description}
			\item[applicationConnectors] Protocol and port o which the application listens (example: \textit{HTTP} and \textit{8080})
			\item[adminConnectors] Protocol and port o which the server statistics are available  (example: \textit{HTTP} and \textit{8081})
		\end{description}
\end{description}

\paragraph*{Login Server}
\begin{description}
	\item[mockToken] Substitute for Google ID token used for testing purposes (example: \textit{loremipsum})
	\item[databaseServerUrl] URL of a Database Server instance (example: \textit{http://localhost:8092})
	\item[server] Jetty web server configuration
	\begin{description}
		\item[applicationConnectors] Protocol and port o which the application listens (example: \textit{HTTP} and \textit{8090})
		\item[adminConnectors] Protocol and port o which the server statistics are available  (example: \textit{HTTP} and \textit{8091})
	\end{description}
\end{description}

\paragraph*{Database Server}
\begin{description}
	\item[mysqlUri] Full Java Database Connectivity URI of MySQL server schema where application data are located (example: \textit{jdbc:mysql://localhost/bachelors})
	\item[mysqlUser] User at MySQL server (example: \textit{root})
	\item[mysqlPass] Pasword of the user MySQL user (example: \textit{password})
	\item[redisUri] Full URL of Redis server (example: \textit{localhost})
	\item[redisPort] Optional port of the Redis server (example: \textit{1467})
	\item[redisPass] Password to the Redis server (example: \textit{password})
	\item[server] Jetty web server configuration
	\begin{description}
		\item[applicationConnectors] Protocol and port o which the application listens (example: \textit{HTTP} and \textit{8092})
		\item[adminConnectors] Protocol and port o which the server statistics are available  (example: \textit{HTTP} and \textit{8093})
	\end{description}
\end{description}

\section{Running the Server}
To start the server simply run the compiled JAR files with argument \textit{server} and path to the configuration file. For example:\\
\textit{java -jar ConnectionServer.jar server config.yml}


	
	\chapter{Contents of Enclosed SD Card}
	
	%change appropriately
	
	\begin{figure}
		\dirtree{%
			.1 readme.txt\DTcomment{the file with SD card contents description}.
			.1 documentation\DTcomment{the directory of documentation}.
			.1 server\DTcomment{the directory of the server}.
			.2 BachelorsPrototype\DTcomment{the directory of server source codes}.
			.2 executables\DTcomment{the directory with executables and configuration files}.
			.1 thesis\DTcomment{the directory of the thesis}.
			.2 latex\DTcomment{the directory of \LaTeX{} source codes of the thesis}.			
			.2 thesis.pdf\DTcomment{the thesis text in PDF format}.
		}
	\end{figure}
	
\end{document}
